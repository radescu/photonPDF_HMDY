\documentclass[pdftex,twocolumn,epjc3]{svjour3}          % twocolumn

\RequirePackage[T1]{fontenc}

\smartqed  % flush right qed marks, e.g. at end of proof

\RequirePackage{graphicx}
\RequirePackage{mathptmx}      % use Times fonts if available on your TeX system
\RequirePackage{flushend}
\RequirePackage[numbers,sort&compress]{natbib}
\RequirePackage[colorlinks,citecolor=blue,urlcolor=blue,linkcolor=blue]{hyperref}
\RequirePackage{amsmath}
\RequirePackage[english]{babel} 
\RequirePackage{bm}              
\RequirePackage{lineno}    
\RequirePackage[latin9]{inputenc}  
  
\journalname{Eur. Phys. J. C}

\begin{document}


\title{\boldmath The photon PDF from high-mass Drell Yan data at the LHC}


%\thankstext[$\star$]{t1}{Thanks to the title}
%\thankstext{e1}{e-mail: magic1@xxx.xx}
%\thankstext{e2}{e-mail: magic2@xxx.xx}

\author{F.~Giuli\thanksref{a}
\and and the xFitter Developers' team: V.~Bertone\thanksref{b,k}
\and D.~Britzger\thanksref{c}
\and S.~Carrazza\thanksref{d}
\and A.~Cooper-Sarkar\thanksref{a}
\and A.~Glazov\thanksref{c}
\and K.~Lohwasser\thanksref{e}
\and A.~Luszczak\thanksref{f}
\and F.~Olness\thanksref{g}
\and R.~Pla\v cakyt\.e\thanksref{h}
\and V.~Radescu\thanksref{a,d}
\and J.~Rojo\thanksref{b,k}
\and R.~Sadykov\thanksref{i}
\and P.~Shvydkin\thanksref{i}
\and O.~Zenaiev\thanksref{c}
\and and M.~Lisovyi\thanksref{j}
}


\institute{University of Oxford,1 Keble Road, Oxford OX1 3NP, United Kingdom \label{a}
\and Department of Physics and Astronomy,  VU University, NL-1081 HV Amsterdam, The Netherlands \label{b}
\and Nikhef Theory Group Science Park 105, 1098 XG Amsterdam, The Netherlands \label{k}
\and DESY Hamburg, Notkestrasse 85 D-22609, Hamburg, Germany \label{c}
\and CERN, CH-1211 Geneva 23, Switzerland \label{d}
\and DESY Zeuthen, Platanenallee 6 D-15738, Zeuthen, Germany \label{e}
\and T.Kosciuszko Cracow University of Technology, 30-084 Cracow, Poland \label{f}
\and SMU Physics, Box 0175 Dallas, TX  75275-0175, United States of America \label{g}
\and Institut f\"ur Theoretische Physik, Universit\"at Hamburg, Luruper Chaussee 149, D--22761 Hamburg, Germany \label{h}
\and Joint Institute for Nuclear Research (JINR), Joliot-Curie 6, 141980, Dubna, Moscow Region, Russia \label{i}
\and Physikalisches Institut, Ruprecht-Karls-Universit\"at Heidelberg, Heidelberg, Germany \label{j}
}






\date{Received: date / Accepted: date}
% The correct dates will be entered by the editor


\maketitle

\begin{abstract}
 Achieving the highest precision for theoretical predictions
  at the LHC requires the 
  calculation of hard-scattering cross-sections that include
  perturbative QCD corrections up to (N)NNLO and electroweak (EW)
  corrections up to NLO.
   %
  Parton distribution functions (PDFs) need to be
  provided with matching accuracy, which in the case of QED effects
  involves introducing the photon parton distribution of the proton,
  $x\gamma(x,Q^2)$.
   %
  In this work a determination of the photon PDF from
  fits to recent ATLAS measurements of high-mass Drell-Yan dilepton
  production at $\sqrt{s}=8$ TeV is presented.
  %
  This analysis is based on the {\tt xFitter} framework,
  and has required improvements both in the {\tt APFEL} program, to account
  for NLO QED effects, and in the {\tt aMCfast} interface to account
  for the photon-initiated contributions in the EW calculations within
  {\tt MadGraph5\_aMC@NLO}.
  %
  The results are compared with other recent QED fits and
  determinations of the photon PDF, consistent results are found.


\end{abstract}


%\usepackage[latin9]{inputenc}
%%\usepackage{color}
%\usepackage{graphicx}
%\usepackage{babel}
%\usepackage{bm}
%\usepackage{amsmath}
%\usepackage{lineno}

%\makeatletter

%%%%%%%%%%%%%%%%%%%%%%%%%%%%%% LyX specific LaTeX commands.                                                                                                                                              
%% Because html converters don't know tabularnewline                                                                                                                                                     
\providecommand{\tabularnewline}{\\}

\def\gsim{\mathrel{\rlap{\lower4pt\hbox{\hskip1pt$\sim$}}
    \raise1pt\hbox{$>$}}}         %greater than or approx. symbol                                                                                                                                        
\def\lsim{\mathrel{\rlap{\lower4pt\hbox{\hskip1pt$\sim$}}
    \raise1pt\hbox{$<$}}}         %less than or approx. symbol                                                                                                                                           

%%%%%%%%%%%%%%%%%%%%%%%%%%%% Textclass specific LaTeX commands.                                                                                                                                        
 % Fix a bug in REVTeX 4.1                                                                                                                                                                               
% \def\lovname{List of Videos}
% \@ifundefined{textcolor}{}
% {%                                                                                                                                                                                                      
%   \definecolor{BLACK}{gray}{0}
%   \definecolor{WHITE}{gray}{1}
%   \definecolor{RED}{rgb}{1,0,0}
%   \definecolor{GREEN}{rgb}{0,1,0}
%   \definecolor{BLUE}{rgb}{0,0,1}
%   \definecolor{CYAN}{cmyk}{1,0,0,0}
%   \definecolor{MAGENTA}{cmyk}{0,1,0,0}
%   \definecolor{YELLOW}{cmyk}{0,0,1,0}
%}
\makeatother
%%%%%%%%%%%%%%%%%%%%%%%%%%%%%%%%%%%%%%%%%%%%%%%%%%                                                                                                                                                       


\newcommand{\la}{\left\langle}
\newcommand{\ra}{\right\rangle}
\newcommand{\lc}{\left[}
\newcommand{\rc}{\right]}
\newcommand{\lp}{\left(}
\newcommand{\rp}{\right)}
\newcommand{\as}{\alpha_s}


%\tableofcontents{}

%%%%%%%%%%%%%%%%%%%%%%%%%%%%%%%%%%%%%%%%%%%%%%%%%%
\section{Introduction}

Precision phenomenology at the LHC requires theoretical calculations
which include not only QCD corrections, where NNLO is rapidly becoming
the standard, but also electroweak (EW) corrections, which are
particularly significant for observables directly sensitive to the TeV
region, where EW Sudakov logarithms are enhanced.
%
An important ingredient of these electroweak corrections is the photon
parton distribution function (PDF) of the proton, $x\gamma(x,Q^2)$, which must be introduced to absorb
the collinear divergences arising in initial-state QED emissions.

The first PDF fit to include both QED corrections and a photon PDF was MRST2004\-QED~\cite{Martin:2004dh}, where the photon PDF was taken from
a model and tested on HERA data for direct photon production.
%
Almost 10 years later, the NNPDF2.3QED analysis~\cite{Ball:2012cx,Ball:2013hta} provided a
first model-independent determination of the photon PDF based on
Drell-Yan (DY) data from the LHC.
%
The resulting photon PDF was however affected by large uncertainties
due to the limited sensitivity of the data used as input to that fit.
%
The determination of $x\gamma(x,Q^2)$ from NNPDF2.3QED was later combined
with the state-of-the-art quark and gluon
PDFs from NNPDF3.0, together with an improved QED evolution,
to construct the NNPDF3.0QED set~\cite{Bertone:2016ume,Ball:2014uwa}.
%
The CT group has also  released a QED fit using a similar
strategy as the MRST2004QED one, named the CT14QED set~\cite{Schmidt:2015zda}.

A recent breakthrough concerning the determination of the photon content of
the proton
has been the realization that  $x\gamma(x,Q^2)$
can be calculated in terms of
inclusive lepton-proton deep-inelastic scattering (DIS) structure functions.
%
The photon PDF resulting from this
strategy is called  LUXqed~\cite{Manohar:2016nzj} and its residual uncertainties are now at the few
percent level, not too different from those
of the quark and gluon PDFs.
%
A related approach by the HKR~\cite{Harland-Lang:2016kog}
group, denoted by HKR16 in the following, also leads to a similar photon PDF
as compared to the LUXqed calculation, although in this case no estimate
for the associated uncertainties is provided.

The aim of this work is to perform a direct determination of the
photon PDF from recent high-mass Drell-Yan measurements from
ATLAS  at $\sqrt{s}=8$ TeV~\cite{Aad:2016zzw}, and to
compare it with some of the existing determinations of $x\gamma(x,Q^2)$
mentioned above.
%
Note that
earlier measurements of high-mass DY from ATLAS and CMS were presented
in Refs.~\cite{CMS:2014jea,Chatrchyan:2013tia,Aad:2013iua}.
%
The ATLAS 8 TeV DY data are provided in terms of both
single-differential cross-section distributions in the dilepton invariant mass,
$m_{ll}$, and of double-differential 
cross-section distributions in $m_{ll}$ and $|y_{ll}|$, the absolute value of rapidity of the
lepton pair, and in $m_{ll}$ and $\Delta\eta_{ll}$, the difference in
pseudo-rapidity between the two leptons.
%
Using the Bayesian reweighting method~\cite{Ball:2011gg,Ball:2010gb}
applied to NNPDF2.3QED, it was shown in the
same publication~\cite{Aad:2016zzw} that these
measurements provided significant information on $x\gamma(x,Q^2)$.

The goal of this study is therefore to investigate further these
constraints from the ATLAS high-mass DY measurements on the photon PDF,
this time by means of a direct PDF fit performed within the
open-source {\tt xFitter} framework~\cite{Alekhin:2014irh}.
%
State-of-the-art theoretical calculations are employed, in particular
the inclusion of  NNLO QCD and NLO QED corrections to the PDF evolution and
the computation of the DIS structure
functions as implemented in the {\tt APFEL} program~\cite{Bertone:2013vaa}.
%
The implementation of NLO QED effects in {\tt APFEL} is
presented here for
the first time.
%
The inclusion of NLO QED evolution effects is cross checked using the independent
{\tt QEDEVOL} code \cite{Sadykov:2014aua} based on the {\tt QCDNUM} evolution program~\cite{Botje:2010ay}.

%
The resulting determination of $x\gamma(x,Q^2)$
represents an important validation test of
recent developments in theory and data concerning
our understanding of the nature and implications
of the photon PDF.

The outline of this paper is as follows.
%
Sect.~\ref{sec:theory} reviews the ATLAS 8 TeV high-mass DY data together
with the theoretical formalism of the DIS and Drell-Yan cross-sections
used in the analysis.
%
Sect.~\ref{sec:fitsettings} presents the settings of the PDF
fit within the {\tt xFitter} framework.
% 
The fit results are then discussed in Sect.~\ref{sec:results}, where
they are compared to determinations by other groups.
%
Finally, Sect.~\ref{sec:conclusions} summarises and discusses the results and
future lines of investigation.
%
Appendix~\ref{sec:appendixAPFEL} contains a detailed
description of the implementation and validation of NLO QED
corrections to the DGLAP PDF evolution equations
and DIS structure functions, which are
available now in {\tt APFEL}.

\section{Theory}
Two processes contribute to opposite sign, same family, dilepton production at the LHC: 
the Drell-Yan quark-antiquark process and the photon-induced process. Both the contributions can be 
simulated with MadGraph5{\_}aMC@NLO (version 2.4.3) and interfaced to APPLgrid (version 01-04-70) and aMCfast (version 01-03-00). A special release of APPLgrid is used to account for the photon PDF within the proton {\it need references for the programmes}.
Both contributions are generated in the 5-flavour scheme, where all the quarks, except for the \textit{top}
 quark, are treated as massless quarks; all the calculations are performed at fixed-order (FO) without 
parton showers. 

Theoretical predictions for both the one-dimensional $\frac{d\sigma}{dm_{ll}}$ distribution 
(where $m_{ll}$ is the invariant mass of the dilepton pair in the final state) and the double-differential 
distributions $\frac{d^{2}\sigma}{dm_{ll}d|y_{ll}|}$ (where $|y_{ll}|$ is the rapidity of the dilepton pair) 
and $\frac{d^{2}\sigma}{dm_{ll}\Delta\eta_{ll}}$ (where $\Delta\eta_{ll}$ represents the difference in 
pseudorapidity between the two leptons) are generated for both the electron and the muon channels.
 
These predictions are generated using the same selections as in reference~\cite{jhep08-2016-009}
as follows:
\begin{itemize}
\item the invariant mass of the lepton pair is required to be greater than 116 GeV;
\item the absolute value of the pseudorapidity of each lepton is required to be less than 2.5;
\item the transverse momentum ($p_{T}$) of the leading lepton has to be greater than 40 GeV;
\item the $p_{T}$ of the sub-leading lepton has to be greater than 30 GeV.
\end{itemize} 
The binning used is the same as used in reference~\cite{jhep08-2016-009}. For the invariant mass 
distribution, there are 12 bins between 116 GeV and 1.5 TeV with variable bin widths; and for both of the 
 the two-dimensional distributions, there are five different histograms, each one for a different invariant
 mass range: (a) 116 GeV < $m_{ll}$ < 150 GeV; (b) 150 GeV < $m_{ll}$ < 200 GeV; (c) 200 GeV < $m_{ll}$ < 300 GeV; (d) 300 GeV < $m_{ll}$ < 500 GeV; (e) 500 GeV < $m_{ll}$ < 1500 GeV.
 The APPLgrids for the first three $m_{ll}$ intervals are divided into 12 bins with fixed bin 
width between $|y_{ll}^{mim}|$ ($|\Delta\eta_{ll}|$)  = 0.0 (0.0) and $|y_{ll}^{max}|$ ($|\Delta\eta_{ll}|$) = 2.4 (3.0), while the final two $m_{ll}$ intervals are divided into 6 bins with fixed bin width scanning the same $|y_{ll}|$ and $|\Delta\eta_{ll}|$ ranges.

Dynamical renormalization ($\mu_{R}$) and factorization ($\mu_{R}$) scales are used in the calculations 
and both are set to $m_{ll}$. The theoretical calculations were validated by comparing both the NLO QCD + LO EW predictions and the 
LO PI predictions to those computed using the FEWZ 3.1 framework. These calculations are evaluated in the $G_{F}$ electroweak scheme, with the following values for the couplings:
 $\alpha_{S}$ = 0.118; $1/\alpha_{EW}$ = 1/127. The difference between the two predictions is at most 1${\%}$, for both the 1-dimensional and the 2-dimensional distributions.

In order to make a next-to-next-to-leading order (NNLO) fit k-factors ($k_{F}$) are computed matching
 the NLO QCD + LO EW cross sections to higher order (HO) calculations. These are computed using 
FEWZ, with the same input parameters as for the NLO computations. The $k_{F}$ are defined as:
\begin{equation}
k_{F}=\frac{NNLO\  QCD  + NLO\  EW \sigma}{NLO\  QCD + LO\  EW \sigma}
\end{equation}
The MMHT2014NNLO PDF set is used to compute both numerator and denominator.
 The $k_{F}$ are close to the unity and their variation is $\sim 2\%$. {\it provide Table of Final k-factors?}


Discuss theory improvements: addition of the NLO QED+QCD piece 

\section{Settings}
\label{sec:fitsettings}

In this section we discuss the settings of
the PDF fit, including the parametrization of the photon PDF
that will be adopted.

In order to make a full PDF fit the  ATLAS Drell-Yan data data are fitted together with the final combined inclusive 
cross section data from HERA~\cite{Abramowicz:2015mha}.
%
The HERA data provide information on the quark/antiquark and gluon content of 
the proton and the Drell-Yan data add information on the photon content of the proton.
%
The NLO and NNLO pQCD predictions are fitted to the data using the xFitter open source pQCD fitting platform~\cite{xFitter}.
The DGLAP equations~\cite{dglap} are solved using the programme APFEL which has been modified to include 
the photon PDF in the proton~\cite{Bertone:2013vaa}.
%
The DGLAP equations yield the PDFs at all scales if they are input as finctions of $x$ at a starting scale $Q^2_0$, which 
should be large enough that perturbative QCD can be assumed to be valid. For the present analysis this value is chosen
to be $Q^2_0 = 7.5~$GeV$^2$.
%
This is also the value chosen for the minimum value of $Q^2$ for data entering the fit.
The charm and beauty masses are chosen to be $m_c=1.47~$GeV and $m_b=4.5~$GeV following the HERA analysis. 
The value of $\alpha_s(M_Z)$ is chosen to be $\alpha_s(M_Z)=0.118$~\cite{PDG}. 
The value of $Q^2_0$ is above the charm mass squared, however a version of the programme 
is used which displaces the charm threshold from the charm mass~\cite{charmthresh} such that the threshold is at $Q^2_0$.
%
The form of the $\chi^2$ used for the fit is that defined in the H1 paper~\cite{h1chisqdef}. 
Alternative forms have also been tried with no significant difference to our results.
 
The PDF parametrisation input at $Q^2_0$ is determined by the technique of saturation of the $\chi^{2}$~\cite{h1chisqsat}.
%
The parametrised PDFs are the valence distributions $xu_{v}$ and $xd_{v}$, the gluon distribution $xg$, and the \textit{u}-type and \textit{d}-type sea, $x\bar{U}$, $x\bar{D}$, where $x\bar{U} = x\bar{u}$ and $x\bar{D} = x\bar{d} + x\bar{s}$, and finally the photon distribution $x\gamma$. The following standard functional form is used to parametrise them:
\begin{equation}
xf(x) = Ax^{B}(1-x)^{C}(1+Dx+Ex^{2})
\end{equation}
where the normalisation parameters $A_{u_{v}}$, $A_{d_{v}}$ and $A_{g}$ are constrained by the number sum-rules and the 
momentum sum-rule, respectively. The \textit{B} parameters $B_{\bar{U}}$ and $B_{\bar{D}}$ are set equal, such that there 
is a single \textit{B} parameter for the sea distribution. The data are not sensitive to the 
strangeness content of the proton which is thus set such that $x\bar{s} = 0.5\bar{D}$, following the ATLAS 
analysis~\cite{Aad:2012sb}. The further constraint $A_{\bar{U}} = 0.5 A_{\bar{D}}$ is imposed such that $\bar{u}=x\bar{d}$ as $x \to 0$.
The \textit{D} and \textit{E} parameters are introduced one by one until no significant 
improvement in $\chi^{2}$ is found. 

 For the NNLO fit a $\chi^{2}/ndf = 1.18$, with a partial $\chi^2/ndp = 1.15$ for the high-mass Drell-yan data [{\it update with final numbers}], is achieved for the following parametrisation, which has 11 parameters for the quarks and gluons and 5 parameters for the photon:
\begin{eqnarray}
xu_v(x) = A_{u_v}x^{B_{u_v}}(1-x)^{C_{u_v}}(1+E_{u_v}x^{2}), \\
xd_v(x) = A_{d_v}x^{B_{d_v}}(1-x)^{C_{d_v}}, \\
x\bar{U}(x) = A_{\bar{U}}x^{B_{\bar{U}}}(1-x)^{C_{\bar{U}}}, \\
x\bar{D}(x) = A_{\bar{D}}x^{B_{\bar{D}}}(1-x)^{C_{\bar{D}}}, \\
xg(x) = A_{g}x^{B_{g}}(1-x)^{C_{g}}(1+E_{g}x^{2}), \\
x\gamma(x) = A_{\gamma}x^{B_{\gamma}}(1-x)^{C_{\gamma}}(1+D_{\gamma}x+E_{\gamma}x^{2}) \\
\end{eqnarray}
The parametrisation for HERA data differs from that of the HERAPDF2.0 PDF since the starting scale $Q^2_0$ is higher and the additional negative term in the gluon parametrisation is not necessary.
Parametrisation and model uncertainties are considered according to the HERAPDF 
procedure~\cite{hera} 
by adding extra terms which make little difference to the $\chi^2$ of the fit, but which can change the shape of the PDFs. Additonal parameters considered are: the extra negative term for the gluon; $D_{u_v}$, $D_{\bar{u}}$ and $E_{\bar{d}}$. Model variations considered are the variation of: 
$m_b$ from 4.25 to 4.75 GeV; $m_c$ from 1.41 to 1.53 GeV;  $Q_0^2$ up to 10 GeV$^2$; 
$Q_{cut}^2$ up to 10 GeV$^2$; the strangeness fraction down to $f_s=0.4$; the value of $\alpha_s(M_Z)$ from 0.116 to 0.120.

\section{Results}

\subsection{Sensitivity}
 show impact of HM DY on PDFs using sensitivity studies based on
pseudo-data, for which we only use the data uncertainties, while central 
value are fixed:
 HERA I+II vs HERA I+II + HMDY --> see the sensitivity plots from the previous email


conclusion: HMDY data has a large impact on photonPDF 



\subsection{PDF Fits}

In order to make a full PDF fit the  ATLAS Drell-Yan data data are fitted together with the final combined inclusive 
cross section data from HERA~\cite{hera}. The HERA data provide information on the quark/antiquark and gluon content of 
the proton and the Drell-Yan data add information on the photon content of the proton. {\it and maybe refine the quark/antiquark content, refer sensitivity study}.   
The NLO and NNLO pQCD predictions are fitted to the data using the xFitter open source pQCD fitting platform~\cite{xFitter}.
The DGLAP equations~\cite{dglap} are solved using the programme QCDNUM which has been modified to include 
the photon PDF in the proton~\cite{qcdnum}.
The DGLAP equations yield the PDFs at all scales if they are input as finctions of $x$ at a starting scale $Q^2_0$, which 
should be large enough that perturbative QCD can be assumed to be valid. For the present analysis this value is chosen
to be $Q^2_0 = 7.5~$GeV$^2$. This is also the value chosen for the minimum value of $Q^2$ for data entering the fit.
The charm and beauty masses are chosen to be $m_c=1.47~$GeV and $m_b=4.5~$GeV following the HERA analysis. 
The value of $\alpha_s(M_Z)$ is chosen to be $\alpha_s(M_Z)=0.118$~\cite{PDG}. 
The value of $Q^2_0$ is above the charm mass squared, however a version of the programme 
is used which displaces the charm threshold from the charm mass~\cite{charmthresh} such that the threshold is at $Q^2_0$.
The form of the $\chi^2$ used for the fit is that defined in the H1 paper~\cite{h1chisqdef}. 
Alternative forms have also been tried with no significant difference to our results.
 
The PDF parametrisation input at $Q^2_0$ is determined by the technique of saturation of the $\chi^{2}$~\cite{h1chisqsat}.
The parametrised PDFs are the valence distributions $xu_{v}$ and $xd_{v}$, the gluon distribution $xg$, and the \textit{u}-type and \textit{d}-type sea, $x\bar{U}$, $x\bar{D}$, where $x\bar{U} = x\bar{u}$ and $x\bar{D} = x\bar{d} + x\bar{s}$, and finally the photon distribution $x\gamma$. The following standard functional form is used to parametrise them:
\begin{equation}
xf(x) = Ax^{B}(1-x)^{C}(1+Dx+Ex^{2})
\end{equation}
where the normalisation parameters $A_{u_{v}}$, $A_{d_{v}}$ and $A_{g}$ are constrained by the number sum-rules and the 
momentum sum-rule, respectively. The \textit{B} parameters $B_{\bar{U}}$ and $B_{\bar{D}}$ are set equal, such that there 
is a single \textit{B} parameter for the sea distribution. The data are not sensitive to the 
strangeness content of the proton which is thus set such that $x\bar{s} = 0.5\bar{D}$, following the ATLAS 
analysis~\cite{atlasstrange}. The further constraint $A_{\bar{U}} = 0.5 A_{\bar{D}}$ is imposed such that $\bar{u}=x\bar{d}$ as $x \to 0$.
The \textit{D} and \textit{E} parameters are introduced one by one until no significant 
improvement in $\chi^{2}$ is found. 

 For the NLO fit a $\chi^{2}/ndf = 1225.3/1084 = 1.13$, with a partial $\chi^2/ndp = 46.9/48$ for the high-mass Drell-yan data [{\it please separate contribution of log term and correlated term in the partial chisq table to allow this calculation to be accurate- I estimated it}], is achieved for the following parametrisation, which has 15 parameters for the quarks and gluons and 5 parameters for the photon:
\begin{eqnarray}
xu_v(x) = A_{u_v}x^{B_{u_v}}(1-x)^{C_{u_v}}(1+D_{u_v}x+E_{u_v}x^{2}), \\
xd_v(x) = A_{d_v}x^{B_{d_v}}(1-x)^{C_{d_v}}, \\
x\bar{U}(x) = A_{\bar{U}}x^{B_{\bar{U}}}(1-x)^{C_{\bar{U}}}(1+D_{\bar{U}}x+E_{\bar{U}}x^2), \\
x\bar{D}(x) = A_{\bar{D}}x^{B_{\bar{D}}}(1-x)^{C_{\bar{D}}}, \\
xg(x) = A_{g}x^{B_{g}}(1-x)^{C_{g}}(1+E_{g}x^{2}), \\
x\gamma(x) = A_{\gamma}x^{B_{\gamma}}(1-x)^{C_{\gamma}}(1+D_{\gamma}x+E_{\gamma}x^{2}) \\
\end{eqnarray}
Figures **%Fig.~\ref{PDF_100GeV}
 show the PDF distributions $x_{u_v},xd_{d_v},x\bar{u}, x\bar{d}, xg$ at $Q^{2}$ = 10$^{2}$ GeV$^{2}$, while  Figures **%Fig.~\ref{PDF_1000GeV} 
show them at $Q^{2}$ = 10$^{4}$ GeV$^{2}$. {\it Redo figures for the just Dubar final parametrisation, conisder adding +Dg and Eubar as parametrisation variations. Also consider model variations  like change of Q2cut, Q20, fs, mc,mb}.
[{\it When showing PDF distributions for the NLO fit at two scales.
show ubar and dbar as well, sbar,cbar,bbar are not interesting in the current context.
Then repeat with this parametrisation at NNLO and quote NNLO partial chisq for DY data. Make plots of NNLo vs NLo only for the photon PDF. 
Here you will almost 
certainly find a worse overall chisq, but it will be worse for HERA which really likes the 
negative gluon term. It will probably not be worse for the DY. We do not need to discuss the HERA NNLO features here}].
%\begin{figure}
%\includegraphics[width=5.14cm]{uv_100.ps} 
%\includegraphics[width=5.14cm]{dv_100.ps} 
%\includegraphics[width=5.14cm]{g_100.ps} 
%\caption{PDF distributions at $Q^{2}$ = 10$^{2}$ GeV$^{2}$: (a) \textit{u} - valence; (b) \textit{d} - valence; (c) gluon}
%\label{PDF_100GeV}
%\end{figure}
%\begin{figure}
%\centering
%\includegraphics[width=5.14cm]{uv_1000}
%\includegraphics[width=5.14cm]{dv_1000} 
%\includegraphics[width=5.14cm]{g_1000} 
%\caption{PDF distributions at $Q^{2}$ = 10$^{4}$ GeV$^{2}$: (a) \textit{u} - valence; (b) \textit{d} - valence; (c) gluon}
%\label{PDF_1000GeV}
%\end{figure}
In these figures comparisons are made to the NNPDF3.0PDF set [{\it use NNPDF3.0 not the rwg version}] and the HERAPDF2.0 set.
One can see that the shape of the $xd_{u_v}$ distribution is close to that of HERAPDF2.0 
because of the dominance of HERA data in the fit. {\it More comments?}

%Fig.~\ref{hmDY_2D} 
Fig. shows the comparison between the hmDY two-dimensional distribution and the predictions. {\it Just make it for the 15 parameter Dubar fit.}
%\begin{figure}
%\centering
%\subfigure[]{\includegraphics[width=5.14cm]{picture_confirmation/hmDY_1}} 
%\subfigure[]{\includegraphics[width=5.14cm]{picture_confirmation/hmDY_2}} 
%\subfigure[]{\includegraphics[width=5.14cm]{picture_confirmation/hmDY_3}} 
%\subfigure[]{\includegraphics[width=5.14cm]{picture_confirmation/hmDY_4}} 
%\subfigure[]{\includegraphics[width=5.14cm]{picture_confirmation/hmDY_5}} 
%\caption{Comparison between $\dfrac{d^{2}\sigma}{dm_{ll}d|y_{ll}|}$ hmDY data and fit results; the various fit differ just on the number of free parameters.}
%\label{hmDY_2D}
%\end{figure}
The $\chi^{2}$ values for each separate fitted dataset and the output parameters from the various fits can be found in Tables * %Fig.~\ref{chi2_scan} 
and in Table *%Fig.~\ref{par_scan} 
respectively. {\it Do this ONLy for the central +Dubar fit}
%\begin{figure}
%\centering
%\subfigure[]{\includegraphics[width=8.44cm]{picture_confirmation/chi2_scan}} 
%\caption{Comparison between $\chi^{2}$ for the different fits just described above.}
%\label{chi2_scan}
%\end{figure}
%\begin{figure}
%\centering
%\subfigure[]{\includegraphics[width=12.14cm]{picture_confirmation/par_scan}} 
%\caption{Comparison between the output parameters of the different fits just described above.}
%\label{par_scan}
%\end{figure}

Focusing now on the photon PDF distribution, there is some impact on the photon PDF from adding parameters to the quark and gluon PDFS, particularly from adding a $D_g$ and an $E_{\bar{U}}$ parameter.
This impact has been included in the parametrisation variations and is shown 
both at the starting scale (7.5 GeV$^{2}$) and at 10$^{4}$ GeV$^{2}$ 
 in Fig *
%Fig.~\ref{photon_scan}. 
%\begin{figure}
%\centering
%\subfigure[]{\includegraphics[width=6.96cm]{picture_confirmation/photon_7_5}} 
%\subfigure[]{\includegraphics[width=6.96cm]{picture_confirmation/photon_10000}} 
%\caption{Comparison between the photon PDF distributions for the different fits just described above: (a) at the starting scale; (b) at the evolved scale.}
%\label{photon_scan}
%\end{figure}

Fig %Fig.~\ref{photon_zoom} 
shows the photon distribution in the range $0.045 < x < 0.35$, region where high mass DY data are most 
sensitive to this quantity. The new fit results 
have an uncertainty between 20{\%} and 30{\%}, which is 
considerably reduced compared to the  NNPDF30qed NLO photon PDF which is also 
shown for comparison. {\it Not the rwg version}.
The predictions for the LUXqed~\cite{luxqed} photon PDF and the 
HKR photon PDF~\cite{hkr}  are shown compared to the NNLO PDF from the present analysis in Fig.** {\it Compare our NNLO PDF when you have it}.
In this kinematic region, the fit
 predictions agree with LUXqed and with the HKR photon PDF at the 1-$\sigma$ level. 


%\begin{figure}
%\centering
%\subfigure[]{\includegraphics[width=6.96cm]{picture_confirmation/photon_7_5_zoom}} 
%\subfigure[]{\includegraphics[width=6.96cm]{picture_confirmation/photon_10000_zoom}} 
%\caption{Comparison between the photon PDF distributions for the different fits just described above: (a) at the starting scale; (b) at the evolved scale.}
%\label{photon_zoom}
%\end{figure}



%\begin{figure*}[th]
%\begin{centering}
%\includegraphics[width=0.75\textwidth]{figs/}
%\par\end{centering}
%\caption{alphas across bottom transition }
%\end{figure*}


\section{Conclusions}

\label{sec:conclusions}

The determination of the photon content of the proton has
attracted a considerable amount of attention recently.
%
In this work, we have presented a determination of the photon PDF from
a PDF fit based on HERA inclusive structure functions and recent ATLAS measurements
of high-mass Drell-Yan cross-sections.
%
We confirm that the high-mass DY data provides significant constraints on the photon PDF
in the region $0.05 \le x \le 0.3$.
%
We find good agreement within uncertainties with two recent calculations of the photon PDF,
LUXqed and HKR for the region of $x$ where the DY data has sensitivity.
%
On the other hand, we also find that a direct determination of the photon PDF
is still far from being competitive with the LUXqed calculation, which uses as input
the high-precision measurements of inclusive structure function of the proton.

The results of this study have been made possible by a number of technical developments
that should be of direct application for future PDF fits accounting for QED corrections,
such as the implementation of $\mathcal{O}\lp alpha^2\rp$ corrections to the DGLAP
evolution and the DIS coefficient functions in {\tt APFEL} or the extension of
{\tt aMCfast} to be able to deal with the photon-initiated calculations provided
by {\tt MadGraph5\_aMC@NLO}.
%
Our results also illustrate the flexibility of the {\tt xFitter} toolbox to extend
its capabilities from the standard quark and gluon PDF fits.


{\bf Acknowledgements}.
%
We thank L. Harland-Lang for providing us a {\tt LHAPDF6} grid
of the HKR photon determination.
%
The work of V.~B., F.~G. and J.~R. has been supported
by the European Research Council Starting Grant ``PDF4BSM".




\appendix
\section{Implementation of NLO QED corrections in APFEL}
\label{sec:appendixAPFEL}

In this appendix, the details of the implementation of the
combined NLO QCD+QED corrections in the {\tt APFEL} program are presented.
%
As discussed in Ref.~\cite{Bertone:2013vaa}, the implementation of the
LO QED corrections to the DGLAP evolution equations includes many
simplifications, in particular the fact that QED and QCD corrections
do not mix and therefore the DGLAP equations, as well as the evolution
equations for the running of the $\alpha_s$ and $\alpha$ couplings,
are decoupled.
%
When increasing the perturbative accuracy to NLO,
this property does not hold anymore, and QED and QCD
contributions mix both in the DGLAP and in the coupling evolution
equations.
%
On top of this complication, QED corrections introduce the presence of
diagrams in which a real photon is present either in the initial or in
the final state, and these have to be included in the computation of the
DIS structure functions.

In the following, the discussion starts by considering how to generalize the equations for
the running of the QCD and QED couplings, finding that the mixed
QCD+QED terms have a negligible impact.
%
Then the extension of the DGLAP evolution equations
to account for the complete NLO QCD+QED effects is discussed.
%
Finally, the modifications introduced by the NLO QED
corrections in both the neutral-current and the charged-current DIS
structure functions are discussed together with those that lead to the 
appearance of photon-initiated
contributions.

\subsection{Evolution of the couplings}

As mentioned above, the NLO QCD+QED corrections induce the presence of
mixed terms in the evolution equations of $\alpha_s$ and $\alpha$.
%
In practice, the QCD $\beta$-function receives corrections
proportional to $\alpha$ and, vice-versa, the QED $\beta$-function
receives corrections proportional to $\alpha_s$, in such a way that
the coupling evolution equations read:
\begin{equation}\label{CoupledEq}
\begin{array}{rcl}
\displaystyle \mu^2\frac{\partial \alpha_s}{\partial \mu^2} &=& \displaystyle
                                                \beta^{\rm QCD}(\alpha_s,\alpha)\,,\\
\\
\displaystyle \mu^2\frac{\partial \alpha}{\partial \mu^2} &=& \displaystyle \beta^{\rm QED}(\alpha_s,\alpha)\,.
\end{array}
\end{equation}
As a consequence, these evolution equations form a set of coupled
differential equations. Up to three loops ($i.e.$ NLO), the
$\beta$-functions can be expanded as:
\begin{equation}
\beta^{\rm QCD}(\alpha_s,\alpha) = -\alpha_s\left[\beta_0^{(\alpha_s)}\left(\frac{\alpha_s}{4\pi}\right)+\beta_1^{(\alpha_s\alpha)}\left(\frac{\alpha_s}{4\pi}\right) \left(\frac{\alpha}{4\pi}\right)+\beta_1^{(\alpha_s^2)}\left(\frac{\alpha_s}{4\pi}\right)^2+\dots\right]\,,
\end{equation}
and:
\begin{equation}
\beta^{\rm QED}(\alpha_s,\alpha) = -\alpha\left[\beta_0^{(\alpha)}\left(\frac{\alpha}{4\pi}\right)+\beta_1^{(\alpha\alpha_s)}\left(\frac{\alpha}{4\pi}\right) \left(\frac{\alpha_s}{4\pi}\right)+\beta_1^{(\alpha^2)}\left(\frac{\alpha}{4\pi}\right)^2+\dots\right]\,.
\end{equation}
where the mixed terms, $\beta_1^{(\alpha_s\alpha)}$ and
$\beta_1^{(\alpha\alpha_s)}$, and the pure NLO QED term,
$\beta_1^{(\alpha^2)}$, can be found in
Ref.~\cite{Surguladze:1996hx}. Taking into account a factor four due
to the different definitions of the expansion parameters, one finds:
\begin{equation}\label{eq:NewBetaTerms}
\beta_1^{(\alpha_s\alpha)} = -2\sum_{i=1}^{n_f}
e_q^2\,\qquad\beta_1^{(\alpha\alpha_s)} = -\frac{16}{3}N_c\sum_{i=1}^{n_f} e_q^2\,,\qquad \beta_1^{(\alpha^2)} = -4\left(n_l+N_c\sum_{i=1}^{n_f} e_q^2\right)\,,
\end{equation}
where $N_c=3$ is the number of colours, $e_q$ is the electric charge
of the quark flavour $q$, and $n_f$ and $n_l$ are the number of active
quark and lepton flavours, respectively.

Eq.~(\ref{CoupledEq}) can be written in the vectorial form:
\begin{equation}\label{CoupledEqVect}
\mu^2\frac{\partial {\bm \alpha}}{\partial \mu^2} = {\bm \beta}\left({\bm \alpha}(\mu)\right)\,,
\end{equation}
with:
\begin{equation}
  {\bm \alpha} = {\alpha_s \choose \alpha}\qquad\mbox{and}\qquad  {\bm \beta} = {\beta^{\rm QCD} \choose \beta^{\rm QED}}\,.
\end{equation}
Eq.~(\ref{CoupledEqVect}) is an ordinary differential equation that
can be numerically solved using, for example, Runge-Kutta methods.

The first two terms in eq.~(\ref{eq:NewBetaTerms}) are responsible for
the coupling of the evolution of $\alpha_s$ and $\alpha$, and thus they
introduce a complication that affects both the implementation and the
performance of the code.
%
One can then ask what is the effect of their
presence and whether their removal makes a substantial difference. 

Fig.~\ref{fig:CouplingEvol} shows the comparison between the
evolution at NLO of both couplings $\alpha_s$ and $\alpha$ including
and excluding the mixed terms in the respective
$\beta$-functions.
%
The evolution is performed between the $Z$ mass
scale $M_Z$ and 10 TeV with 5 active quark flavours and 3 active
lepton flavours and uses as boundary conditions
$\alpha_s(M_Z) = 0.118$ and $\alpha(M_Z) = 1/128$.
%
The two curves in
Fig.~\ref{fig:CouplingEvol} are normalised to the respective curves
without mixed terms. It is clear that the mixed terms lead to tiny
relative differences that are at most of $\mathcal{O}(10^{-4})$ at 10
TeV for $\alpha_s$ and $\mathcal{O}(10^{-3})$ at the same scale for
$\alpha$.
%
Thus it is safe to conclude that the mixed terms in the $\beta$-functions
have a negligible effect on the evolution of the couplings and thus they are 
excluded to simplify the code  and to improve the performance
without introducing any significant inaccuracy.

%%%%%%%%%%%%%%%%%%%%%%%%%%%%%%%%%%%%%%%%%%%%%%%%%%%%%%%%
\begin{figure}[h]
\includegraphics[width=6cm,angle=270]{figs/couplings.pdf} 
\caption{Comparison between the running with the scale
$Q$ of the QCD and QED couplings,
  $\alpha_s$ and $\alpha$, including or not the mixed terms in
  the corresponding $\beta$-functions.
%
  The curves are normalised to the result of the respective coupling
  running without the mixed terms included in the $\beta$ functions.}
\label{fig:CouplingEvol}
\end{figure}
%%%%%%%%%%%%%%%%%%%%%%%%%%%%%%%%%%%%%%%%%%%%%%%%%%%%%%%%

\subsection{PDF evolution with NLO QED corrections}

Next the implementation of the full NLO QCD+QED corrections to the
DGLAP evolution equations is considered.
%
The discussion is limited to consideration of
the photon, leptons are not considered.
%
The first step towards an efficient implementation of the solution of
the DGLAP equations in the presence of QED corrections is the adoption
of a suitable PDF basis that diagonalises the splitting function
matrix, decoupling as many equations as possible.
%
Such a basis was introduced in the Appendix A of
Ref.~\cite{Bertone:2015lqa} and will be used also here.
%
Excluding the lepton PDFs, this basis contains 14 independent PDF
combinations and reads:
\begin{equation}\label{eq:EvolBasis}
\begin{array}{ll}
\mbox{\texttt{ 1} : }g & \\
\mbox{\texttt{ 2} : }\gamma & \\
\mbox{\texttt{ 3} : }\displaystyle \Sigma = \Sigma_u + \Sigma_d & \quad
\mbox{\texttt{9} : }\displaystyle V =V_u +  V_d\\
\mbox{\texttt{ 4} : } \displaystyle \Delta_\Sigma = \Sigma_u - \Sigma_d& \quad\displaystyle 
\mbox{\texttt{10} : } \Delta_V = V_u - V_d\\
\mbox{\texttt{ 5} : }T_1^u = u^+ - c^+ &\quad \mbox{\texttt{11} : }V_1^u = u^- - c^- \\
\mbox{\texttt{ 6} : }T_2^u = u^+ + c^+ - 2t^+ &\quad \mbox{\texttt{12} : }V_2^u = u^- + c^- - 2t^-\\
\mbox{\texttt{ 7} : }T_1^d = d^+ - s^+ &\quad \mbox{\texttt{13} : }V_1^d = d^- - s^- \\
\mbox{\texttt{ 8} : }T_2^d = d^+ + s^+ - 2b^+ &\quad \mbox{\texttt{14}
                                               : }V_2^d = d^- + s^- -
                                               2b^-\\
\end{array}
\end{equation}
where $q^\pm = q\pm\overline{q}$ with
$q = u,d,s,c,b,t$. In addition:
\begin{equation}
\begin{array}{ll}
\Sigma_u = u^++c^++t^+, &\quad V_u = u^-+c^-+t^-,\\
\\
\Sigma_d = d^++s^++b^+,&\quad V_d = d^-+s^-+b^-\,.
\end{array}
\end{equation}

The second step is the construction of the splitting function matrix that
determines the evolution of each of the combinations listed in
Eq.~(\ref{eq:EvolBasis}).
%
To this end, the splitting function matrix $P$ is split into a pure
QCD term $\widetilde{P}$, which only depends on $\alpha_s$, and a
mixed QCD+QED correction term $\overline{P}$, which instead contains
contributions proportional to at least one power of the QED coupling
$\alpha$.
%
In practice, this means that
\begin{equation}
P = \widetilde{P} + \overline{P}\,,
\end{equation}
where the pure QCD term reads:
\begin{equation}\label{eq:PureQCDSplittings}
\widetilde{P} = \alpha_s \mathcal{P}^{(1,0)} + \alpha_s^2 \mathcal{P}^{(2,0)}+\dots\, ,
\end{equation}
while the term containing the QED coupling is given by:
\begin{equation}\label{eq:QCD+QEDSplittings}
\overline{P} = \alpha \mathcal{P}^{(0,1)} + \alpha_s\alpha \mathcal{P}^{(1,1)}+\alpha^2 \mathcal{P}^{(0,2)} + \dots \, .
\end{equation}
Note that in the r.h.s. of Eqs.~(\ref{eq:PureQCDSplittings})
and~(\ref{eq:QCD+QEDSplittings}) the convention of
Refs.~\cite{deFlorian:2015ujt,deFlorian:2016gvk} is followed to indicate the power
of $\alpha_s$ and $\alpha$ that each splitting function multiplies.

The structure of the pure QCD splitting function matrix
$\widetilde{P}$ as well as the first term in $\overline{P}$, which
represents the pure LO QED correction, were already discussed in
Ref.~\cite{Bertone:2015lqa}.
%
It is now necessary to analyse the structure of the two additional
terms, namely $\mathcal{P}^{(1,1)}$ and $\mathcal{P}^{(0,2)}$.
%
Starting with the $\mathcal{O}(\alpha_s\alpha)$ term,
%
at this perturbative order, the resulting evolution equations read:
\begin{equation}
\begin{array}{rcl}
\displaystyle\left.\mu^2\frac{\partial}{\partial \mu^2}
\begin{pmatrix}
g\\
\gamma\\
\Sigma\\
\Delta_\Sigma
\end{pmatrix}
\right|_{\mathcal{O}(\alpha_s \alpha)} &=& \displaystyle \begin{pmatrix}
e_\Sigma^2 \mathcal{P}^{(1,1)}_{gg}      & e_\Sigma^2 \mathcal{P}^{(1,1)}_{g\gamma} & \eta^+\mathcal{P}^{(1,1)}_{gq} & \eta^-\mathcal{P}^{(1,1)}_{gq} \\
e_\Sigma^2 \mathcal{P}^{(1,1)}_{\gamma g} & e_\Sigma^2 \mathcal{P}^{(1,1)}_{\gamma\gamma} & \eta^+\mathcal{P}^{(1,1)}_{\gamma q} &\eta^-\mathcal{P}^{(1,1)}_{\gamma q} \\
2 e_\Sigma^2 \mathcal{P}^{(1,1)}_{qg}    & 2 e_\Sigma^2 \mathcal{P}^{(1,1)}_{q\gamma} & \eta^+\mathcal{P}^{+(1,1)}  & \eta^-\mathcal{P}^{+(1,1)}\\
2 \delta_e^2 \mathcal{P}^{(1,1)}_{qg} & 2 \delta_e^2 \mathcal{P}^{(1,1)}_{q\gamma} &\eta^-\mathcal{P}^{+(1,1)} &\eta^+\mathcal{P}^{+(1,1)}
\end{pmatrix}\otimes
\begin{pmatrix}
g\\
\gamma\\
\Sigma\\
\Delta_\Sigma
\end{pmatrix}
\end{array}\,,
\end{equation}

\begin{equation}
\displaystyle\left.\mu^2\frac{\partial}{\partial \mu^2}
\begin{pmatrix}
V\\
\Delta_V
\end{pmatrix} \right|_{\mathcal{O}(\alpha_s \alpha)}= 
\begin{pmatrix}
\eta^+\mathcal{P}^{-(1,1)} & \eta^-\mathcal{P}^{-(1,1)} \\
\eta^-\mathcal{P}^{-(1,1)} & \eta^+\mathcal{P}^{-(1,1)} 
\end{pmatrix}\otimes
\begin{pmatrix}
V\\
\Delta_V
\end{pmatrix}\,,
\end{equation}

\begin{equation}
\begin{array}{ll}
\begin{array}{rcl}
\displaystyle \left.\mu^2\frac{\partial T^u_{1,2}}{\partial \mu^2}\right|_{\mathcal{O}(\alpha_s \alpha)} &=&
\displaystyle e_u^2\mathcal{P}^{+(1,1)}\otimes T^u_{1,2}
\end{array}\,, &
\begin{array}{rcl}
\displaystyle \left.\mu^2\frac{\partial T^d_{1,2}}{\partial \mu^2}\right|_{\mathcal{O}(\alpha_s \alpha)} &=&
\displaystyle e_d^2\mathcal{P}^{+(1,1)} \otimes T^d_{1,2}
\end{array}\,,
\\
\\
\begin{array}{rcl}
\displaystyle \left.\mu^2\frac{\partial V^u_{1,2}}{\partial \mu^2}\right|_{\mathcal{O}(\alpha_s \alpha)} &=&
\displaystyle e_u^2\mathcal{P}^{-(1,1)} \otimes V^u_{1,2}
\end{array}\,, &
\begin{array}{rcl}
\displaystyle \left.\mu^2\frac{\partial V^d_{1,2}}{\partial \mu^2}\right|_{\mathcal{O}(\alpha_s \alpha)} &=&
\displaystyle e_d^2\mathcal{P}^{-(1,1)}\otimes V^d_{1,2}
\end{array}\,.
\end{array}
\end{equation}
where $\otimes$ indicates the Mellin convolution and where 
\begin{equation}
\begin{array}{rcl}
e_{\Sigma}^{2}& \equiv &\displaystyle
N_c(n_ue_{u}^{2}+n_de_{d}^{2})\,,\\
\\
\delta_e^2 & \equiv &\displaystyle N_c(n_u e_u^2 -n_d e_d^2)\,,\\
\\
\eta^{\pm} & \equiv & \displaystyle \frac{1}{2}\left(e_{u}^{2}\pm
  e_{d}^{2}\right)\,,\\
\end{array}
\end{equation}
with $e_u$ and $e_d$ the electric charges of the up- and down-type
quarks, and $n_u$ and $n_d$ the number of up- and down-type active
quark flavours (such that $n_u+n_d=n_f$).
%
%Once expressed this basis, it is possible to include the
%$\mathcal{O}(\alpha_s\alpha)$ corrections in the solution of the DGLAP
%QCD+QED equations in APFEL as done in Ref.~\cite{Bertone:2015lqa}.

Next the $\mathcal{O}(\alpha^2)$ corrections as considered.
%
The expressions of the splitting functions at this order have been
presented in Ref.~\cite{deFlorian:2016gvk}.
%
There are two relevant new features that distinguish these corrections
from the $\mathcal{O}(\alpha)$ and the $\mathcal{O}(\alpha_s\alpha)$
ones.
%
The first one is that, contrary to the other cases in which the
electric charges appears to the second power at most, here they appear
up to the fourth power.
%
As a consequence, new couplings must be introduced:
\begin{equation}
\begin{array}{l}
e_{\Sigma}^4 = N_c(n_{u} e_u^4 + n_{d} e_d^4)\,,\\
\\
\delta_e^4 = N_c(n_{u} e_u^4 - n_{d} e_d^4)\,.
\end{array}
\end{equation}
The second feature is that the dependence on the electric charges of
some of the $\mathcal{O}(\alpha^2)$ splitting functions is not
factorisable as was the case for all the $\mathcal{O}(\alpha)$ and
$\mathcal{O}(\alpha_s\alpha)$ ones and therefore a distinction must be made
 between up- and down-type splitting functions.
%
Taking into account these features, it is possible to show that the
$\mathcal{O}(\alpha^2)$ contributions to the DGLAP equations take the
following form:
\begin{equation}
%\begin{array}{rcl}
\begin{array}{c}
\displaystyle\left.\mu^2\frac{\partial}{\partial \mu^2}
\begin{pmatrix}
g\\
\gamma\\
\Sigma\\
\Delta_\Sigma
\end{pmatrix}
  \right|_{\mathcal{O}(\alpha^2)} =\\
\\
 \displaystyle \frac12\begin{pmatrix}
    0 & 0 & 0 & 0 \\
    0 & 2e_\Sigma^4 \mathcal{P}_{\gamma\gamma}^{(0,2)} & e_u^4 \mathcal{P}_{\gamma
      u}^{(0,2)} + e_d^4 \mathcal{P}_{\gamma d} & e_u^4 \mathcal{P}_{\gamma u}^{(0,2)} - e_d^4 \mathcal{P}_{\gamma d}^{(0,2)}\\
    0 & 4 e_\Sigma^4 \mathcal{P}^{(0,2)}_{q\gamma} &
    e_u^4\mathcal{P}_{uu}^{+(0,2)}
    +e_d^4\mathcal{P}_{dd}^{+(0,2)}+2\eta^+e_\Sigma^2\mathcal{P}^{S(0,2)}_{qq} & e_u^4\mathcal{P}_{uu}^{+(0,2)}-e_d^4\mathcal{P}_{dd}^{+(0,2)} + 2\eta^-e_\Sigma^2\mathcal{P}^{S(0,2)}_{qq}\\
    0 & 4 \delta_e^4 \mathcal{P}^{(0,2)}_{q\gamma} & e_u^4\mathcal{P}_{uu}^{+(0,2)}
    -e_d^4\mathcal{P}_{dd}^{+(0,2)}+2\eta^-\delta_e^2
    \mathcal{P}^{S(0,2)}_{qq} & e_u^4\mathcal{P}_{uu}^{+(0,2)}+e_d^4\mathcal{P}_{dd}^{+(0,2)} + 2\eta^+\delta_e^2 \mathcal{P}^{S(0,2)}_{qq}
\end{pmatrix}\otimes
\begin{pmatrix}
g\\
\gamma\\
\Sigma\\
\Delta_\Sigma
\end{pmatrix}\,,
\end{array}
\end{equation}

\begin{equation}
\displaystyle\left.\mu^2\frac{\partial}{\partial \mu^2}
\begin{pmatrix}
V\\
\Delta_V
\end{pmatrix} \right|_{\mathcal{O}(\alpha^2)}= \frac12
\begin{pmatrix}
e_u^4\mathcal{P}_{uu}^{-(0,2)}+e_d^4\mathcal{P}_{dd}^{-(0,2)} & e_u^4\mathcal{P}_{uu}^{-(0,2)}-e_d^4\mathcal{P}_{dd}^{-(0,2)} \\
e_u^4\mathcal{P}_{uu}^{-(0,2)}-e_d^4\mathcal{P}_{dd}^{-(0,2)} & e_u^4\mathcal{P}_{uu}^{-(0,2)}+e_d^4\mathcal{P}_{dd}^{-(0,2)} 
\end{pmatrix}\otimes
\begin{pmatrix}
V\\
\Delta_V
\end{pmatrix}\,,
\end{equation}

\begin{equation}
\begin{array}{ll}
\begin{array}{rcl}
\displaystyle \left.\mu^2\frac{\partial T^u_{1,2}}{\partial \mu^2}\right|_{\mathcal{O}(\alpha^2)} &=&
\displaystyle e_u^4\mathcal{P}_{uu}^{+(0,2)}\otimes T^u_{1,2}
\end{array}\,, &
\begin{array}{rcl}
\displaystyle \left.\mu^2\frac{\partial T^d_{1,2}}{\partial \mu^2}\right|_{\mathcal{O}(\alpha^2)} &=&
\displaystyle e_d^4\mathcal{P}_{dd}^{+(0,2)} \otimes T^d_{1,2}
\end{array}\,,
\\
\\
\begin{array}{rcl}
\displaystyle \left.\mu^2\frac{\partial V^u_{1,2}}{\partial \mu^2}\right|_{\mathcal{O}(\alpha^2)} &=&
\displaystyle e_u^4\mathcal{P}_{uu}^{-(0,2)} \otimes V^u_{1,2}
\end{array}\,, &
\begin{array}{rcl}
\displaystyle \left.\mu^2\frac{\partial V^d_{1,2}}{\partial \mu^2}\right|_{\mathcal{O}(\alpha^2)} &=&
\displaystyle e_d^4\mathcal{P}_{dd}^{-(0,2)}\otimes V^d_{1,2}
\end{array}\,.
\end{array}
\end{equation}
It should be noted that, as compared to the expressions for
$\mathcal{P}^{(0,2)}$ presented in Ref.~\cite{deFlorian:2016gvk},
the electric charges haven been factored out in such a way that the
expressions of the splitting functions are either independent from the
electric charges or depend on them only through the ratio
$e_\Sigma^2/e_q^2$.
%
%As before, once the $\mathcal{O}(\alpha^2)$ corrections to the QCD+QED
%combined evolution equations are expressed in this form, it is
%possible to include them in {\tt APFEL} and to solve them using the
%same numerical techniques as for the other cases.

As an illustration, the effects of the
$\mathcal{O}(\alpha_s\alpha)$ and $\mathcal{O}(\alpha^2)$ corrections
to the DGLAP evolution equations on the $\gamma\gamma$ luminosity are quantified at
$\sqrt{s} = 13$ TeV. This luminosity is defined as:
\begin{equation}\label{eq:GammaGammaLumi}
  \mathcal{L}_{\gamma\gamma}(M_X) = \frac1{s}\int_{M_X^2/s}^1
  \frac{dx}{x} \gamma(x,M_X^2) \gamma\left(\frac{M_X^2}{xs},M_X^2\right)\,,
\end{equation}
as a function of the final state invariant mass $M_X$.
%
Fig.~\ref{fig:GammaGammaLumi} illustrates the behaviour of
$\mathcal{L}_{\gamma\gamma}$ computed using the photon PDF from the
NNPDF30QED NLO set as an input at $Q_0 = 1$ GeV and evolved to $Q=M_X$
including, on top of the pure QCD NLO evolution, the following
corrections:
\begin{itemize}
\item the $\mathcal{O}(\alpha)$ corrections only,
\item same as above, adding also the mixed
  $\mathcal{O}(\alpha_s\alpha)$ corrections, and
\item the complete NLO QCD+QED corrections accounting for the
  $\mathcal{O}(\alpha+\alpha_s\alpha+\alpha^2)$ effects.
\end{itemize}
The results are shown normalised to the predictions obtained with LO
QED corrections only.
%
It is clear that the $\mathcal{O}(\alpha_s\alpha)$ and
$\mathcal{O}(\alpha^2)$ corrections have a small but non-negligible
impact on the $\gamma\gamma$-luminosity. In particular, these
corrections suppress $\mathcal{L}_{\gamma\gamma}$ by around 10\% at
relatively small values of $M_X$, while the suppression gradually
shrinks to 1-2\% as $M_X$ increases. As expected, most of this effect
comes from the $\mathcal{O}(\alpha_s\alpha)$ corrections, while the
impact of the $\mathcal{O}(\alpha^2)$ ones is substantially smaller.
The $\mathcal{O}(\alpha_s\alpha)$ and $\mathcal{O}(\alpha^2)$
corrections to the DGLAP evolution have more recently been implemented
in the {\tt QEDEVOL} package~\cite{Sadykov:2014aua} based on the
{\tt QCDNUM} evolution code~\cite{Botje:2010ay}. {\tt APFEL} and {\tt
  QEDEVOL} have been found to be in excellent agreement.

%%%%%%%%%%%%%%%%%%%%%%%%%%%%%%%%%%%%%%%%%%%%%%%%%%%%%%%%
\begin{figure}[t]
\includegraphics[width=10cm]{figs/lumi_13tev.pdf} 
\caption{The photon-photon PDF luminosity $\mathcal{L}_{\gamma\gamma}$ at $\sqrt{s} = 13$ TeV as a
  function of the final state invariant mass $M_X$.
  %
  The results with the photon evolved
  with only the $\mathcal{O}(\alpha)$ corrections
  are compared with the corresponding results taking into account the 
  $\mathcal{O}(\alpha+\alpha_s\alpha)$ corrections
  and the complete
  $\mathcal{O}(\alpha+\alpha_s\alpha+\alpha^2)$ effects,
  normalised in all three cases to the $\mathcal{O}(\alpha)$ result.
  %
  The calculation has been performed using the central value of the NNPDF3.0QED NLO
  fit.  }
\label{fig:GammaGammaLumi}
\end{figure}
%%%%%%%%%%%%%%%%%%%%%%%%%%%%%%%%%%%%%%%%%%%%%%%%%%%%%%%%

\subsection{DIS structure functions}

When considering NLO QCD+QED corrections to the DIS structure
functions, it becomes necessary to include into the hard cross
sections all the $\mathcal{O}(\alpha)$ diagrams where one photon is
either in the initial state or emitted from an incoming quark (or
possibly an incoming lepton).
%
Such diagrams, being of purely QED origin, have associated coefficient
functions that can be easily derived from the QCD expressions by
properly adjusting the colour factors.
%
This correspondence holds irrespective of whether mass effects are
included.

The main complication of the inclusion of these corrections arises
from their flavour structure. In fact, in the case of
quarks the isospin symmetry is broken due to the fact that the
coupling of the photon is proportional to the squared charge of the
parton to which it couples (a quark or a lepton).
%
In the following, the neutral-current (NC)
case, where lepton and proton exchange a neutral boson $\gamma^*/Z$,
and the charged-current (CC) case, where instead lepton and proton
exchange a charged $W$ boson, are addressed separately.
%

First  the $\mathcal{O}(\alpha)$
contributions to a generic NC structure function $F$ are considered.
%
Due to the fact that to this order there is no mixing between QCD and
QED, such corrections can easily be derived from the
$\mathcal{O}(\alpha_s)$ coefficient functions just by adjusting the
colour factors by setting $C_F=T_R=1$ and $C_A=0$.
%
Referring, $e.g.$, to the expressions reported in
Ref.~\cite{Ellis:1991qj}, the coefficient functions become:
\begin{equation}\label{eq:alphaCFs}
\begin{array}{rcl}
\displaystyle C_{i;q}^{(\alpha)} &=& \displaystyle \frac{C_{i;q}^{(\alpha_s)}}{C_F}\\
\\
\displaystyle C_{i;\gamma}^{(\alpha)} &=& \displaystyle \frac{C_{i;g}^{(\alpha_s)}}{T_R}
\end{array}\qquad i = 2,L,3\,.
\end{equation}
In order to construct the corresponding structure
functions, considering that the coupling between a photon and a quark
of flavour $q$ is proportional to $e_q^2$, one also needs to adjust
the electroweak couplings should be adjusted as follows:
 \begin{equation}
\begin{array}{rcl}
\widetilde{B}_q &=& B_qe_q^2\quad\mbox{for}\quad F_2,F_L\,, \\
\\
\widetilde{D}_q &=& D_qe_q^2\quad\mbox{for}\quad F_3\,, \\
\end{array}
\end{equation}
where $B_q$ and $D_q$ are defined, $e.g.$, in
Ref.~\cite{Adloff:2003uh}.
%
Following this prescription, it is possible to write the
$\mathcal{O}(\alpha)$ contributions to the NC structure functions as:
\begin{equation}
\begin{array}{rcl}
F_{2,L}^{{\rm NC},(\alpha)} &=& \displaystyle x \sum_{q} \widetilde{B}_q\left[C_{2,L;q}^{(\alpha)}\otimes
(q+\overline{q}) + C_{2,L;\gamma}^{(\alpha)} \otimes \gamma
                         \right]\,,\\
\\
xF_3^{{\rm NC},(\alpha)} &=& \displaystyle x \sum_{q} \widetilde{D}_q\left[C_{3;q}^{(\alpha)}\otimes
(q-\overline{q}) + C_{3;\gamma}^{(\alpha)} \otimes \gamma
                         \right]\,.
\end{array}
\end{equation}
This structure holds for both massless and massive
structure functions. This aspect is relevant to the construction of
the FONLL general-mass structure functions.

For the CC case the procedure to obtain the
expressions of the $\mathcal{O}(\alpha)$ coefficient functions is
exactly the same as in the NC case (see
Eq.~(\ref{eq:alphaCFs})).
%
However, this case is more complicated
because the flavour structure of CC structure functions is more
complex.
%
Taking into account the presence of a factor $e_q^2$ every time that a
quark of flavour $q$ couples to a photon, the $\mathcal{O}(\alpha)$
corrections to the CC structure functions $F_2$ and $F_L$ for the
production of a neutrino or an anti-neutrino take the form:
\begin{equation}\label{compactNu}
\begin{array}{rcl}
F_{2,L}^{{\rm CC},\nu,(\alpha)} &=& \displaystyle
                              x\sum_{U=u,c,t}\sum_{D=d,s,b}|V_{UD}|^2\left[C_{2,L;q}^{(\alpha)}\otimes\left(e_D^2D +e_U^2\overline{U}\right) +2 C_{2,L;\gamma}^{(\alpha)}\otimes\gamma\right]\,,\\
\\
F_{2,L}^{{\rm CC},\overline{\nu},(\alpha)} &=& \displaystyle
x\sum_{U=u,c,t}\sum_{D=d,s,b}|V_{UD}|^2\left[C_{2,L;q}^{(\alpha)}\otimes\left(e_D^2\overline{D}
    +e_U^2U\right) +2 C_{2,L;\gamma}^{(\alpha)}\otimes\gamma\right]\,,
\end{array}
\end{equation}
where $V_{UD}$ are the elements of the CKM matrix.
%
The flavour structure of $F_3$ is instead slightly different:
\begin{equation}\label{compactNuF3}
\begin{array}{rcl}
xF_3^{{\rm CC},\nu,(\alpha)} &=& \displaystyle
                              x\sum_{U=u,c,t}\sum_{D=d,s,b}|V_{UD}|^2\left[C_{3;q}^{(\alpha)}\otimes\left(e_D^2D -e_U^2\overline{U}\right) +2 C_{3;\gamma}^{(\alpha)}\otimes\gamma\right]\,,\\
\\
xF_3^{{\rm CC},\overline{\nu},(\alpha)} &=& \displaystyle
x\sum_{U=u,c,t}\sum_{D=d,s,b}|V_{UD}|^2\left[C_{3;q}^{(\alpha)}\otimes\left(-e_D^2\overline{D}
    +e_U^2U\right) +2 C_{3;\gamma}^{(\alpha)}\otimes\gamma\right]\,.
\end{array}
\end{equation}
In order to simplify the implementation, it is advantageous to assume
that, in these particular corrections, the CKM matrix is a $3 \times 3$
unitary matrix. Note however that the exact CKM matrix is still used
in the QCD part of the structure functions.
%
This approximation introduces an inaccuracy of the order of the QED
coupling $\alpha$ times the value of the off-diagonal elements of the
CKM matrix and therefore it is numerically negligible.

As an illustration of the impact of the $\mathcal{O}(\alpha)$
correction on the DIS structure functions, Fig.~\ref{fig:StructFuncs}
shows the effect of introducing these contributions on top of the pure
QCD computation at NLO.
%
The plots are produced by evolving the
NNPDF3.0QED NLO set from $Q_0=1$ GeV to $Q=100$ GeV including the full
NLO QCD+QED corrections discussed in the previous section and using
the resulting evolved PDFs to compute the NC (left panel) and the CC
(right panel) DIS structure functions in the FONLL-B scheme, including
the $\mathcal{O}(\alpha)$ corrections to the coefficient functions
discussed above.
%
The predictions are shown normalised to the pure QCD computation where
the QED corrections are absent both in the evolution and in the
computation of the structure functions.

%%%%%%%%%%%%%%%%%%%%%%%%%%%%%%%%%%%%%%%%%%%%%%%%%%%%%%%%
\begin{figure}[t]
\includegraphics[width=6cm,angle=270]{figs/NLOQEDCorrections_NC.pdf}
\includegraphics[width=6cm,angle=270]{figs/NLOQEDCorrections_CC.pdf}
\caption{The effects of the NLO QED corrections on the neutral-current
(left) and charged-current (right) DIS structure functions
$F_2, F_L$ and $xF_3$, normalised to the pure QCD results.
%
The calculation has been performed in the FONLL-B general-mass scheme using the
central NNPDF3.0QED NLO
set as input.
%
Note that QED effects enter both via DGLAP evolution and the
$\mathcal{O}(\alpha)$ DIS coefficient functions.
%
The behaviour of $xF_3$ in the right plot for $x\sim 0.007$ is explained
by the fact that this structure function exhibits a node in that region.
}
\label{fig:StructFuncs}
\end{figure}
%%%%%%%%%%%%%%%%%%%%%%%%%%%%%%%%%%%%%%%%%%%%%%%%%%%%%%%%

It is clear that the impact of the full NLO QCD+QED corrections is
pretty small especially in the low-$x$ region where it is well below
1\%.
%
In the large-$x$ region, instead, the presence of a
photon-initiated contribution has a more significant effect because of
the suppression of the QCD distributions (quarks and gluon) relative
to the photon PDF and the impact of the QED corrections reaches the
2\% level.
%
It should be stressed that the behaviour around
$x=10^{-2}$ of the CC $xF_3$ (green curve in the right panel) is driven
by a change of sign of the predictions (in other words,
$xF_3$ exhibits a node in this region) so that the ratio diverges.

%%%%%%%%%%%%%%%%%%%%%


\bibliographystyle{spphys}       % APS-like style for physics
\bibliography{main}   % name your BibTeX data base

%\bibliographystyle{JHEP}
%\bibliography{main}

%%%%%%%%%%%%%%%%%%%%%%%%%%%%%%%%%%%%%%%%%%%%%%%%%%

\end{document}
