\section{Introduction}

Precision phenomenology at the LHC requires theoretical calculations
which include not only QCD corrections, where NNLO is rapidly becoming
the standard, but also electroweak (EW) corrections, which are
particularly significant for observables directly sensitive to the TeV
region, where EW Sudakov logarithms are enhanced.
%
An important ingredient of these electroweak corrections is the photon
parton distribution function (PDF) of the proton, $x\gamma(x,Q^2)$, which must be introduced to absorb
the collinear divergences arising in initial-state QED emissions.

The first PDF fit to include both QED corrections and a photon PDF was
MRST2004QED~\cite{Martin:2004dh}, where the photon PDF was taken from
a model and tested on HERA data for direct photon production.
%
Almost 10 years later, the NNPDF2.3QED analysis~\cite{Ball:2012cx,Ball:2013hta} provided a
first model-independent determination of the photon PDF based on
Drell-Yan (DY) data from the LHC.
%
The resulting photon PDF was however affected by large uncertainties
due to the limited sensitivity of the data used as input to that fit.
%
The determination of $x\gamma(x,Q^2)$ from NNPDF2.3QED was later combined
with the state-of-the-art quark and gluon
PDFs from NNPDF3.0, together with an improved QED evolution,
to construct the NNPDF3.0QED set~\cite{Bertone:2016ume,Ball:2014uwa}.
%
The CT group has also  released a QED fit using a similar
strategy as the MRST2004QED one, named the CT14QED set~\cite{Schmidt:2014aba}.

A recent breakthrough concerning the determination of the photon content of
the proton
has been the realisation that  $x\gamma(x,Q^2)$
can be calculated in terms of
inclusive lepton-proton deep-inelastic scattering (DIS) structure functions.
%
The residual uncertainties in the photon PDF resulting from this
strategy, dubbed LUXqed~\cite{Manohar:2016nzj}, are now at the few
percent level, not too different from those
of the quark and gluon PDFs.
%
A related approach by the HKR~\cite{Harland-Lang:2016apc}
group, denoted by HKR16 in the following, also leads to a similar photon PDF
as compared to the LUXqed calculation, although in this case no estimate
for the associated uncertainties is provided.

The aim of this work is to perform a direct determination of the
photon PDF from recent high-mass Drell-Yan measurements from
ATLAS  at $\sqrt{s}=8$ TeV~\cite{Aad:2016zzw}, and to
compare it with some of the existing determinations of $x\gamma(x,Q^2)$
mentioned above.
%
Note that
earlier measurements of high-mass DY from ATLAS and CMS were presented
in Refs.~\cite{CMS:2014jea,Chatrchyan:2013tia,Aad:2013iua}.
%
The ATLAS 8 TeV DY data is provided in terms of both
single-differential distributions in the dilepton invariant mass,
$m_{ll}$, and of double-differential
cross sections in $m_{ll}$ and $|y_{ll}|$, the rapidity of the
lepton pair, and in $m_{ll}$ and $\Delta\eta_{ll}$, the difference in
pseudo-rapidity between the two leptons.
%
Using the Bayesian reweighting method~\cite{Ball:2011gg,Ball:2010gb}
applied to NNPDF2.3QED, it was shown in the
same publication~\cite{Aad:2016zzw} that these
measurements provided significant information on $x\gamma(x,Q)$.

The goal of this study is therefore to investigate further these
constraints from the ATLAS high-mass DY measurements on the photon PDF,
this time by means of a direct PDF fit performed within the
open-source {\tt xFitter} framework~\cite{Alekhin:2014irh}.
%
State-of-the-art theoretical calculations are employed, in particular
we include NNLO QCD and NLO QED corrections to the PDF evolution and
the computation of the DIS structure
functions as implemented in the {\tt APFEL} program~\cite{Bertone:2013vaa}.
%
The implementation of NLO QED effects in {\tt APFEL} is
presented here for
the first time.
%
The resulting determination of $x\gamma(x,Q)$
represents an important validation test of
recent developments in theory and data concerning
our understanding of the nature and implications
of the photon PDF.

The outline of this paper is as follows.
%
Sect.~\ref{sec:theory} reviews the ATLAS 8 TeV high-mass DY data together
with the theoretical formalism of the DIS and Drell-Yan cross-sections
used in the analysis.
%
In Sect.~\ref{sec:fitsettings} we present the settings of the PDF
fit within the {\tt xFitter} framework.
% 
The fit results are then discussed in Sect.~\ref{sec:results}, where
they are compared to determinations by other groups.
%
Finally, in Sect.~\ref{sec:conclusions} we summarise the results and
discuss future lines of investigation.
%
Appendix~\ref{sec:appendixAPFEL} contains a detailed
description of the implementation and validation of NLO QED
corrections to the DGLAP PDF evolution equations
and DIS structure functions,
available now in {\tt APFEL}.
